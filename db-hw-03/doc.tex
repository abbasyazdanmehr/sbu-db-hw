\documentclass[a4paper]{article}

%%% usepackage %%%
\usepackage{color}
\usepackage{ulem}
\usepackage{amsmath}
\usepackage{amssymb}
\usepackage{xfrac}
\usepackage{import}
\usepackage{listings}
\usepackage{hyperref}
\usepackage[inner=2cm, outer=2cm, top=2cm, bottom=2cm]{geometry}
\usepackage{xepersian}
\settextfont{Dubai}
\fontsize{30}{12}\selectfont
\usepackage{eso-pic}% http://ctan.org/pkg/eso-pic
\usepackage{lipsum}% http://ctan.org/pkg/lipsum

%%% define color %%%
\definecolor{codegreen}{rgb}{0,0.6,0}
\definecolor{codegray}{rgb}{0.5,0.5,0.5}
\definecolor{codepurple}{rgb}{0.58,0,0.82}
\definecolor{backcolour}{rgb}{0.95,0.95,0.92}

\definecolor{dkgreen}{rgb}{0,0.6,0}
\definecolor{gray}{rgb}{0.5,0.5,0.5}
\definecolor{mauve}{rgb}{0.58,0,0.82}

\def\ojoin{\setbox0=\hbox{$\bowtie$}%
  \rule[-.02ex]{.25em}{.4pt}\llap{\rule[\ht0]{.25em}{.4pt}}}
\def\leftouterjoin{\mathbin{\ojoin\mkern-5.8mu\bowtie}}
\def\rightouterjoin{\mathbin{\bowtie\mkern-5.8mu\ojoin}}
\def\fullouterjoin{\mathbin{\ojoin\mkern-5.8mu\bowtie\mkern-5.8mu\ojoin}}

\lstset{frame=tb,
  language=SQL,
  aboveskip=3mm,
  belowskip=3mm,
  showstringspaces=false,
  columns=flexible,
  basicstyle={\small\ttfamily},
  numbers=none,
  numberstyle=\tiny\color{gray},
  keywordstyle=\color{blue},
  commentstyle=\color{dkgreen},
  stringstyle=\color{mauve},
  breaklines=true,
  breakatwhitespace=true,
  tabsize=3
}


%%% newcommand %%%
\newcommand{\emailone}{\texttt{abbas.yazdanmehr1@gmail.com}}
\newcommand{\fulltitle}[2]{\title{#1 \\ #2}}
\newcommand{\myinf}{
	\author{\noindent
عباس یزدان مهر
\\
99243077\\
 مهندسی کامپیوتر, دانشگاه شهید بهشتی
\\
\emailone
	}
}
\newcommand{\goodby}{\begin{center}{\huge
پایان
}\end{center}}



\begin{document}

\fulltitle{
پایگاه داده
}{
تمرین سوم
}

\myinf

\maketitle

\newpage


\noindent \myinf


\section{}
S (تهیه کنندگان) \\
P (قطعات) \\
J (پروژه ها) \\
\subsection*{a}

\begin{displaymath}
  \sigma_{city='tehran'}(J)
\end{displaymath}

\subsection*{b}

\begin{displaymath}
  \Pi_{sname}(\sigma_{p\#='p2'}(S \bowtie SPJ))
\end{displaymath}

\subsection*{c}

\begin{displaymath}
  \Pi_{sname}(((\sigma_{color='blue'}(P)) \bowtie SPJ) \bowtie S)
\end{displaymath}


\subsection*{d}

\begin{displaymath}
  \Pi_{sname}(((\Pi_{s\#, p\#}(SPJ)) \div (\Pi_{p\#}(P))) \bowtie S)
\end{displaymath}

\subsection*{e}

\begin{displaymath}
  \Pi_{sname}(S) - \Pi_{sname}(\sigma_{p\#='p2'}(S \bowtie SPJ))
\end{displaymath}

\subsection*{f}

\begin{latin}
\begin{displaymath}
  \Pi_{p\#}(SPJ \bowtie (\sigma_{city='tehran'}(S))) \cup \Pi_{p\#}(SPJ \bowtie (\sigma_{city='tehran'}(J)))
\end{displaymath}  
\end{latin}

\newpage
\myinf
%%% Q2
\section{}
\subsection*{a}

\begin{displaymath}
  \sigma_{amount>1000} (loan)
\end{displaymath}

\subsection*{b}

\begin{displaymath}
  \Pi_{loan\_number}(\sigma_{amount>1000} (loan))
\end{displaymath}

\subsection*{c}

\begin{displaymath}
  \Pi_{customer\_name} (borrower) \cap \Pi_{customer\_name} (depositor)
\end{displaymath}


\subsection*{d}

\begin{displaymath}
  \Pi_{customer\_name} (\sigma_{branch\_name = 'centeral'} (loan \bowtie borrower))
\end{displaymath}


\subsection*{e}

\begin{displaymath}
  \Pi_{customer\_name} (\sigma_{branch\_name = 'b1'} (loan \bowtie borrower))
\end{displaymath}

\subsection*{f}

\begin{latin}
  $\Pi_{customer\_name} ( $\\
  \indent \indent $ [\sigma_{branch\_city = 'tehran'} (branch) \bowtie \Pi_{account\_number, branch\_name}(account)] $ \\
  \indent \indent \indent \indent $ \bowtie \Pi_{customer\_name, account\_number}(depositor) $ \\
  $)$
  
\end{latin}

\newpage
\myinf
%%% Q3
\section{}
\begin{latin} $\theta$ join: \end{latin}
\begin{displaymath}
  R \bowtie_\theta S = \sigma_\theta (R \times S)
\end{displaymath}


\begin{latin}\noindent natural join: \end{latin}
\begin{displaymath}
  R \bowtie S = \{r \cup s | r \in R \land s \in S \land Fun(r \cup s)\}
\end{displaymath}

که $Fun(x)$ در آن یک محمول است که بررسی می کند که آیا
$x$ یک تابع است یا خیر.

\newpage
\myinf
%%% Q4
\section{}
\subsection*{a}
\begin{displaymath}
  \{<a>|\exists b, c (<a,b,c> \in r_1)\}
\end{displaymath}

\subsection*{b}
\begin{displaymath}
  \{<a, b, c>|<a,b,c> \in r_1 \land b=17\}
\end{displaymath}

\subsection*{c}
\begin{displaymath}
  \{<a, b, c>|<a, b, c> \in r_1 \lor <a, b, c> \in r_2\}
\end{displaymath}

\subsection*{d}
\begin{displaymath}
  \{<a, b, c>|<a, b, c> \in r_1 \land <a, b, c> \in r_2\}
\end{displaymath}

\subsection*{e}
\begin{displaymath}
  \{<a, b, c>|<a, b, c> \in r_1 \land <a, b, c> \notin r_2\}
\end{displaymath}

\subsection*{f}
\begin{displaymath}
  \{<a, b, c>|\exists c (<a,b,c> \in r_1) \lor \exists a (<a,b,c> \in r_2)\}
\end{displaymath}

\newpage
\goodby
\end{document}
